\documentclass[a4paper,11pt]{book}
\synctex=1
\usepackage[english]{babel}
\usepackage[T1]{fontenc}
\usepackage[utf8]{inputenc}
\usepackage{textcomp}
\usepackage{stmaryrd}
\usepackage{algorithm2e}
\usepackage{natbib}
\usepackage{cite}
\usepackage[noend]{algpseudocode}
\usepackage{amsthm}
\usepackage{array}
\usepackage{xcolor}
\usepackage{dsfont}
\usepackage{eurosym}
\usepackage{yfonts}
%\usepackage[demo]{graphicx}
\usepackage{caption}
\usepackage{subcaption}

\usepackage{amsmath,amssymb,amsfonts,graphicx,shorttoc,textpos,caption,here,titlesec}
\usepackage{verbatim,enumerate,hyperref,dsfont,fancyhdr,setspace,array}
\usepackage[margin=3cm]{geometry}
\usepackage{fancyvrb}
\usepackage{xcolor}
\usepackage{listings}
\lstloadlanguages{R}
\usepackage{booktabs}
\usepackage{color}

\usepackage{cleveref}

\definecolor{mygreen}{rgb}{0,0.6,0}
\definecolor{mymauve}{rgb}{0.58,0,0.82}
\lstset{literate=%
{é}{{\'e}}1
{è}{{\`e}}1
{à}{{\o}}1
{Æ}{{\AE}}1
{Å}{{\AA}}1
{Ø}{{\O}}1
}
\lstset{extendedchars=\true}
\lstset{inputencoding=ansinew}

\newcommand{\pen}{\text{pen}}

\newtheoremstyle{custom}% name of the style to be used
  {}% measure of space to leave above the theorem. E.g.: 3pt
  {}% measure of space to leave below the theorem. E.g.: 3pt
  {}% name of font to use in the body of the theorem
  {0pt}% measure of space to indent
  {\scshape}% name of head font
  {\quad}% punctuation between head and body
  { }% space after theorem head; " " = normal interword space
  {\thmname{#1}\thmnumber{ #2}\thmnote{ (#3)}}
  
\theoremstyle{custom}

\newsavebox{\BBbox}
\newenvironment{DDbox}[1]{
\begin{lrbox}{\BBbox}\begin{minipage}{\linewidth}}
{\end{minipage}\end{lrbox}\noindent{\usebox{\BBbox}} \\
[.5cm]}
\newtheorem{thm}{Theorem}[section]
\newtheorem{de}{Definition}[section]
\newtheorem{lm}{Lemma}[section]
\newtheorem{pr}{Proposition}[section]
\newtheorem{as}{Assumption}[section]
\newtheorem{pro}{Proof}[section]
\newtheorem{rmk}{Remark}[section]
\newtheorem{rem}{Reminder}[section]
\newtheorem{nota}{Notations}[section]
\renewcommand\thefootnote{{footnote}}
\addto\captionsenglish{
\renewcommand{\contentsname}
{Table des matières}
\renewcommand{\listfigurename}{Table des figures}
}

\graphicspath{{Illustration/}}


\hypersetup{colorlinks=true,linkcolor=blue}

\linespread{1.1}

\parindent0cm

\DeclareMathOperator*{\argmin}{arg\,min}
%\DeclareMathOperator*{\liminf}{lim\,inf}
%\DeclareMathOperator*{\limsup}{lim\,sup}
\titleformat{\chapter}[display]
{\normalfont\Large\filcenter\sffamily}
{\rule[1mm]{15mm}{1.5mm} \hspace{4mm} \large{\chaptertitlename} \thechapter \hspace{5mm} \rule[1mm]{15mm}{1.5mm}}
{1pc}
{\titlerule
\vspace{1pc}
\huge}

\begin{document}

\title{Regression on graph by spectral decomposition\\
\large cut-off and aggregation}
\author{
\begin{tabular}{ccccc}
	Alexander Kreiß & Xavier Loizeau\\
\end{tabular}
}
\date{}

\maketitle

\chapter{Introduction}
\section{The data}\label{I}

\section{Likelihood}


\chapter{Bayesian approach with hierarchical sieve prior}
\section{Sieve priors}

\section{Hierarchical priors}

\chapter{Frequentist estimation in circular deconvolution}
\section{Empirical distribution and projection estimates}

\subsection{Estimating the Fourier coefficients individually}
\subsection{Estimation of the Fourier sequence}

\subsection{Penalized contrast model selection}

\section{Aggregation estimator}

\subsection{Estimating the Fourier coefficients individually}
\subsection{Estimating the Fourier sequence}


\chapter{Generalization of graph regression}
\section{Dependent data}
\subsection{Projection estimator and model selection}
\subsection{Aggregation estimator}

\section{Partially unknown operator}
\subsection{Projection estimator and model selection}
\subsection{Aggregation estimator}

\bibliography{biblio.bib}{}
\nocite{*}
\bibliographystyle{apalike}


\end{document}
%%% Local Variables:
%%% mode: latex
%%% TeX-master: t
%%% End:
